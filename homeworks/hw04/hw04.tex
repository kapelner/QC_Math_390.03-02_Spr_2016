\documentclass[12pt]{article}

\include{preamble}

\newtoggle{professormode}
\toggletrue{professormode} %STUDENTS: DELETE or COMMENT this line



\title{MATH 390.03-02 / 650 Fall 2015 Homework \#4}

\author{Professor Adam Kapelner} %STUDENTS: write your name here

\iftoggle{professormode}{
\date{Due 4PM in my mail slot, Friday, March 4, 2016 \\ \vspace{0.5cm} \small (this document last updated \today ~at \currenttime)}
}

\renewcommand{\abstractname}{Instructions and Philosophy}

\begin{document}
\maketitle

\iftoggle{professormode}{
\begin{abstract}
The path to success in this class is to do many problems. Unlike other courses, exclusively doing reading(s) will not help. Coming to lecture is akin to watching workout videos; thinking about and solving problems on your own is the actual ``working out.''  Feel free to \qu{work out} with others; \textbf{I want you to work on this in groups.}

Reading is still \textit{required}. For this homework set, read about Bayesian Hypothesis testing, Bayes Factors and again the beta prior, the binomial-beta bayesian formulation, the beta-binomial model as well as mixture models and kernels of PMFs / PDFs. Also read ch8-10 in McGrayne.

The problems below are color coded: \ingreen{green} problems are considered \textit{easy} and marked \qu{[easy]}; \inorange{yellow} problems are considered \textit{intermediate} and marked \qu{[harder]}, \inred{red} problems are considered \textit{difficult} and marked \qu{[difficult]} and \inpurple{purple} problems are extra credit. The \textit{easy} problems are intended to be ``giveaways'' if you went to class. Do as much as you can of the others; I expect you to at least attempt the \textit{difficult} problems. 

Problems marked \qu{[MA]} are for the masters students only (those enrolled in the 650 course). For those in 390, doing these questions will count as extra credit.

This homework is worth 100 points but the point distribution will not be determined until after the due date. See syllabus for the policy on late homework.

Up to 10 points are given as a bonus if the homework is typed using \LaTeX. Links to instaling \LaTeX~and program for compiling \LaTeX~is found on the syllabus. You are encouraged to use \url{overleaf.com}. If you are handing in homework this way, read the comments in the code; there are two lines to comment out and you should replace my name with yours and write your section. The easiest way to use overleaf is to copy the raw text from hwxx.tex and preamble.tex into two new overleaf tex files with the same name. If you are asked to make drawings, you can take a picture of your handwritten drawing and insert them as figures or leave space using the \qu{$\backslash$vspace} command and draw them in after printing or attach them stapled.

The document is available with spaces for you to write your answers. If not using \LaTeX, print this document and write in your answers. I do not accept homeworks which are \textit{not} on this printout. Keep this first page printed for your records.

\end{abstract}

\thispagestyle{empty}
\vspace{1cm}
NAME: \line(1,0){380}
\clearpage
}

\problem{These are questions about McGrayne's book, chapters 8-10.}

\begin{enumerate}

\easysubproblem{When was experimentation introduced to medical science and who introduced it? Are you surprised that it was this recent?}\spc{1}

\easysubproblem{Sir Ronald A. Fisher, the founder of modern experiments, did not believe cigarettes caused lung cancer. What were his two hypotheses for the cause of lung cancer?}\spc{2}

\easysubproblem{Who invented, and what are Bayes Factors? (p116)}\spc{2}

\easysubproblem{Trick question: who convinced Cornfield to stop smoking?}\spc{2}

\easysubproblem{Why were frequentists at a loss to estimate the probability of a nuclear bomb being detonated by accident?}\spc{2}

\easysubproblem{What is \href{https://en.wikipedia.org/wiki/Cromwell\%27s_rule}{Cromwell's Rule}? And, when applying this principle to a Bayesian model what would it imply? (See the Wikipedia link and p123).}\spc{2}

\easysubproblem{Did Bayesian Statistics prevent nuclear accidents? Discuss.}\spc{5}

\easysubproblem{What is the main reason why there are so many variations of Bayesian interpretation? (p129)}\spc{2}

\easysubproblem{What is a large practical drawback of Bayesian inference? (See mid-end of chapter 8).}\spc{2}

\end{enumerate}


\problem{We will again be looking at the beta-prior, binomial-likelihood Bayesian model. Consider $\Xoneton \exchdist \bernoulli{\theta}$ and $\theta \sim \stdbetanot$.}


\begin{enumerate}

\intermediatesubproblem{Design a prior where you believe $\expe{\theta} = 0.5$ and you feel as if your belief represents information contained in five coin flips. Reference 4(i) on HW\#3.}\spc{6}

\intermediatesubproblem{Calculate a 95\% a priori credible region for $\theta$. Use \texttt{R} on your computer (or \href{http://www.r-fiddle.org/}{R-Fiddle} online) and its \texttt{qbeta} function.}\spc{1}

\easysubproblem{You flip the same coin 100 times and you observe 39 heads. Find the distribution of $\theta~|~X$.}\spc{0.5}


\easysubproblem{Calculate a 95\% a posteriori credible region for $\theta$.}\spc{1}

\easysubproblem{Why is your answer to (d) a smaller interval than (b)?}\spc{3}

\intermediatesubproblem{Test the hypothesis that this coin is fair given prior information from (a) and the data from (c). Make sure you say whether you retain or reject the null and justify why.}\spc{10}

\intermediatesubproblem{Calculate the Bayesian $p$-val that the coin is unfair given the intermediate calculation performed in (f).}\spc{3}

\intermediatesubproblem{Test the hypothesis that this coin has a bias towards Heads (not tails) given prior information from (a) and the data from (c). Make sure you say whether you retain or reject the null and justify why.}\spc{6}

\easysubproblem{Calculate the Bayesian $p$-val for the test in (h).}\spc{3}

\easysubproblem{Let's say you wanted to test whether the coin is fair but you are indifferent to any $\theta$ which is different from 0.5 by a margin of 0.1. Write out the hypotheses for this test.}\spc{3}

\intermediatesubproblem{Test the hypotheses from (j) given prior information from (a) and the data from (c). Make sure you say whether you retain or reject the null and justify why.}\spc{10}


\easysubproblem{Calculate the Bayesian $p$-val for the test in (k).}\spc{3}

\intermediatesubproblem{Given the hypotheses from (j), write the formula for the Bayes factor by specifying what $\mathcal{M}_1$ and $\mathcal{M}_2$ are and then writing the integral expression. Do not solve.}\spc{4}


\intermediatesubproblem{[MA] Calculate the Bayes Factor numberically by solving the integral expression in (m). Interpret your value of $K$ (or $B$) according to \href{https://en.wikipedia.org/wiki/Bayes_factor}{wikipedia page about Bayes Factors}.}\spc{9}


\hardsubproblem{Instead of the hypotheses in (j), use the hypotheses $H_0: \theta = 0.5$ vs. $H_a$ being the prior from (a) and calculate the Bayes Factor. Interpret your value of $K$ (or $B$) according to \href{https://en.wikipedia.org/wiki/Bayes_factor}{wikipedia page about Bayes Factors}.}\spc{10}


\intermediatesubproblem{Use a frequentist approach to test the null that this coin is fair given the data from (c). Calculate a $p$-val as well.}\spc{9}

\hardsubproblem{Write about why your answers to all these hypothesis tests (except the one sided test in h) differ from each other and differ from the frequentist test in (p).}\spc{10}

\easysubproblem{Given prior information from (a) and the data from (c), what is the distribution of one future coin flip?}\spc{1}

\intermediatesubproblem{[MA] Rederive the posterior predictive distribution for the general $\theta \sim \stdbetanot$ case for $n$ data points and $m$ future observations.}\spc{9}

\easysubproblem{Given prior information from (a) and the data from (c), what is the distribution of ten future coin flips?}\spc{2}

\easysubproblem{How can you think about the beta-binomial distribution in terms of binomial sampling from bags?}\spc{7}

\intermediatesubproblem{For your distribution in (t) what is the expected number of heads and the standard error number of heads? (See \href{https://en.wikipedia.org/wiki/Beta-binomial_distribution}{wikipedia}).}\spc{7}

\hardsubproblem{[MA] For your distribution in (s), find a 95\% central interval for the number of heads and the number of tails expected on 10 future coin flips.}\spc{12}

\intermediatesubproblem{Given prior information from (a) and the now new data: three tails, find the frequentist and Bayesian confidence intervals for $\theta$.}\spc{2}

\easysubproblem{Forget the prior information from (a). If you get 3 tails, what is the estimate of the probability of heads using the \qu{law of succession}?}\spc{2}

\intermediatesubproblem{Prove that the $\thetahatmmse$ is a shrinkage estimator.}\spc{6}

\easysubproblem{Assume again the prior information from (a). What is the shrinkage proportion $\rho$ for this prior when  estimating $\theta$ via $\thetahatmmse$?.}\spc{2}



\hardsubproblem{Prove that $\thetahatmmse$ is a biased estimator (i.e. its expectation is \textit{not} $\theta$).}\spc{5}

\easysubproblem{Prove that $\displaystyle \limitn \rho = 0$ and therefore this bias $\rightarrow 0$ as your dataset gets larger.}\spc{4}

\hardsubproblem{[MA] Why on Earth should anyone use shrinkage estimators if they're biased? Google it. Discuss.}\spc{6}

\end{enumerate}


\problem{Some quick question on mixture distributions.}

\begin{enumerate}

\easysubproblem{Let $X$ be $\normnot{0}{1^2}$ 1/3 of the time and $\exponential{3}$ 2/3 of the time. What is its pdf?}\spc{6}


\hardsubproblem{Let's say $X \sim \betanot{1}{\beta}$ where $\beta \sim \exponential{\lambda}$. Write an integral expression which when solved, finds the compound / marginal density of $X$. DO NOT solve.}\spc{6}

\hardsubproblem{[MA] Let's say $X \sim \normnot{\theta}{\sigsq}$ where $\theta \sim \normnot{\mu}{\tausq}$. Write an integral expression which when solved, finds the compound / marginal density of $X$. DO NOT solve.}\spc{6}

\end{enumerate}

\problem{We will now have lots of examples finding kernels from common distributions. Some of these questions are silly, but they will force you to think hard about what the kernel is under different situations.}

\begin{enumerate}

\easysubproblem{What is the kernel of $X~|~\theta,~n \sim \binomial{n}{\theta}$?}\spc{3}

\hardsubproblem{What is the kernel of $X, n~|~\theta \sim \binomial{n}{\theta}$? Be careful...}\spc{3}

\easysubproblem{What is the kernel of $X~|~\theta \sim \poisson{\theta}$?}\spc{3}

\easysubproblem{What is the kernel of $X~|~\theta \sim \poisson{\theta}$?}\spc{3}

\hardsubproblem{What is the kernel of $\theta~|~X \sim \poisson{\theta}$? Be careful...}\spc{3}

\easysubproblem{What is the kernel of $X~|~\alpha,~\beta \sim \stdbetanot$?}\spc{6}

\easysubproblem{What is the kernel of $X~|~\theta \sim \exponential{\theta}$?}\spc{3}

\easysubproblem{What is the kernel of $X~|~\theta,~\sigsq \sim \normnot{\theta}{\sigsq}$?}\spc{3}

\hardsubproblem{What is the kernel of $\theta,~\sigsq~|~X \sim \normnot{\theta}{\sigsq}$? Be careful...}\spc{3}

\intermediatesubproblem{What is the kernel of 

\beqn
X~|~N,~\theta,~n \sim \hypergeometric{N}{\theta}{n} := {{{ \theta \choose x} {{N-\theta} \choose {n-x}}}\over {N \choose n}}
\eeqn

where $N$ is the number of total balls in the bag, $\theta$ is the number of success balls in the bag and $n$ is the number drawn out of the bag?}\spc{6}


\end{enumerate}

\end{document}

