\documentclass[12pt]{article}

\include{preamble}

\newtoggle{professormode}
\toggletrue{professormode} %STUDENTS: DELETE or COMMENT this line



\title{MATH 390.03-02 / 650 Fall 2015 Homework \#5}

\author{Professor Adam Kapelner} %STUDENTS: write your name here

\iftoggle{professormode}{
\date{Due 4PM in my mail slot, Friday, March 11, 2016 \\ \vspace{0.5cm} \small (this document last updated \today ~at \currenttime)}
}

\renewcommand{\abstractname}{Instructions and Philosophy}

\begin{document}
\maketitle

\iftoggle{professormode}{
\begin{abstract}
The path to success in this class is to do many problems. Unlike other courses, exclusively doing reading(s) will not help. Coming to lecture is akin to watching workout videos; thinking about and solving problems on your own is the actual ``working out.''  Feel free to \qu{work out} with others; \textbf{I want you to work on this in groups.}

Reading is still \textit{required}. For this homework set, read about mixture models, kernels of PMFs / PDFs, Jeffrey's priors, improper priors, Bayesian model checking. No readings in McGrayne this week; use the time to study for the exam.

The problems below are color coded: \ingreen{green} problems are considered \textit{easy} and marked \qu{[easy]}; \inorange{yellow} problems are considered \textit{intermediate} and marked \qu{[harder]}, \inred{red} problems are considered \textit{difficult} and marked \qu{[difficult]} and \inpurple{purple} problems are extra credit. The \textit{easy} problems are intended to be ``giveaways'' if you went to class. Do as much as you can of the others; I expect you to at least attempt the \textit{difficult} problems. 

Problems marked \qu{[MA]} are for the masters students only (those enrolled in the 650 course). For those in 390, doing these questions will count as extra credit.

This homework is worth 100 points but the point distribution will not be determined until after the due date. See syllabus for the policy on late homework.

Up to 10 points are given as a bonus if the homework is typed using \LaTeX. Links to instaling \LaTeX~and program for compiling \LaTeX~is found on the syllabus. You are encouraged to use \url{overleaf.com}. If you are handing in homework this way, read the comments in the code; there are two lines to comment out and you should replace my name with yours and write your section. The easiest way to use overleaf is to copy the raw text from hwxx.tex and preamble.tex into two new overleaf tex files with the same name. If you are asked to make drawings, you can take a picture of your handwritten drawing and insert them as figures or leave space using the \qu{$\backslash$vspace} command and draw them in after printing or attach them stapled.

The document is available with spaces for you to write your answers. If not using \LaTeX, print this document and write in your answers. I do not accept homeworks which are \textit{not} on this printout. Keep this first page printed for your records.

\end{abstract}

\thispagestyle{empty}
\vspace{1cm}
NAME: \line(1,0){380}
\clearpage
}



\problem{Some quick question on mixture distributions.}

\begin{enumerate}

\easysubproblem{Let $X$ be $\normnot{0}{1^2}$ 1/3 of the time and $\exponential{3}$ 2/3 of the time. What is its pdf?}\spc{6}


\hardsubproblem{Let's say $X~|~\beta \sim \betanot{1}{\beta}$ where $\beta~|~\lambda \sim \exponential{\lambda}$. Write an integral expression which when solved, finds the compound / marginal density of $X$. DO NOT solve.}\spc{6}

\hardsubproblem{[MA] Let's say $X~|~\theta,~\sigsq \sim \normnot{\theta}{\sigsq}$ where $\theta~|~\mu,~\tausq \sim \normnot{\mu}{\tausq}$. Write an integral expression which when solved, finds the compound / marginal density of $X$. DO NOT solve.}\spc{6}

\end{enumerate}

\problem{We will now have lots of examples finding kernels from common distributions. Some of these questions are silly, but they will force you to think hard about what the kernel is under different situations.}

\begin{enumerate}

\easysubproblem{What is the kernel of $X~|~\theta,~n \sim \binomial{n}{\theta}$?}\spc{3}

\hardsubproblem{What is the kernel of $X, n~|~\theta \sim \binomial{n}{\theta}$? Be careful...}\spc{3}

\easysubproblem{What is the kernel of $X~|~\theta \sim \poisson{\theta}$?}\spc{3}

\easysubproblem{What is the kernel of $X~|~\theta \sim \poisson{\theta}$?}\spc{3}

\hardsubproblem{What is the kernel of $\theta~|~X \sim \poisson{\theta}$? Be careful...}\spc{3}

\easysubproblem{What is the kernel of $X~|~\alpha,~\beta \sim \stdbetanot$?}\spc{6}

\easysubproblem{What is the kernel of $X~|~\theta \sim \exponential{\theta}$?}\spc{4}

\easysubproblem{What is the kernel of $X~|~\theta,~\sigsq \sim \normnot{\theta}{\sigsq}$?}\spc{4}

\hardsubproblem{What is the kernel of $\theta,~\sigsq~|~X \sim \normnot{\theta}{\sigsq}$? Be careful...}\spc{4}

\intermediatesubproblem{What is the kernel of 

\beqn
X~|~N,~\theta,~n \sim \hypergeometric{N}{\theta}{n} := {{{ \theta \choose x} {{N-\theta} \choose {n-x}}}\over {N \choose n}}
\eeqn

where $N$ is the number of total balls in the bag, $\theta$ is the number of success balls in the bag and $n$ is the number drawn out of the bag?}\spc{6}


\end{enumerate}

\end{document}

