\documentclass[12pt]{article}

\include{preamble}

\newtoggle{professormode}
\toggletrue{professormode} %STUDENTS: DELETE or COMMENT this line



\title{MATH 390.03-02 / 650 Spring 2016 Homework \#10}

\author{Professor Adam Kapelner} %STUDENTS: write your name here

\iftoggle{professormode}{
\date{Due 11:59PM, Wednesday, May 18, 2016 \\ (but can be handed in without penalty until May 25, 2016 at 8:30AM) \\ \vspace{0.5cm} \small (this document last updated \today ~at \currenttime)}
}

\renewcommand{\abstractname}{Instructions and Philosophy}

\begin{document}
\maketitle

\iftoggle{professormode}{
\begin{abstract}
The path to success in this class is to do many problems. Unlike other courses, exclusively doing reading(s) will not help. Coming to lecture is akin to watching workout videos; thinking about and solving problems on your own is the actual ``working out.''  Feel free to \qu{work out} with others; \textbf{I want you to work on this in groups.}

Reading is still \textit{required}. For this homework set, read about ridge regression and Gibbs sampling. Also read ch17 and the epilogue in McGrayne.

The problems below are color coded: \ingreen{green} problems are considered \textit{easy} and marked \qu{[easy]}; \inorange{yellow} problems are considered \textit{intermediate} and marked \qu{[harder]}, \inred{red} problems are considered \textit{difficult} and marked \qu{[difficult]} and \inpurple{purple} problems are extra credit. The \textit{easy} problems are intended to be ``giveaways'' if you went to class. Do as much as you can of the others; I expect you to at least attempt the \textit{difficult} problems. 

Problems marked \qu{[MA]} are for the masters students only (those enrolled in the 650 course). For those in 390, doing these questions will count as extra credit.

This homework is worth 100 points but the point distribution will not be determined until after the due date. See syllabus for the policy on late homework.

Up to 10 points are given as a bonus if the homework is typed using \LaTeX. Links to instaling \LaTeX~and program for compiling \LaTeX~is found on the syllabus. You are encouraged to use \url{overleaf.com}. If you are handing in homework this way, read the comments in the code; there are two lines to comment out and you should replace my name with yours and write your section. The easiest way to use overleaf is to copy the raw text from hwxx.tex and preamble.tex into two new overleaf tex files with the same name. If you are asked to make drawings, you can take a picture of your handwritten drawing and insert them as figures or leave space using the \qu{$\backslash$vspace} command and draw them in after printing or attach them stapled.

The document is available with spaces for you to write your answers. If not using \LaTeX, print this document and write in your answers. I do not accept homeworks which are \textit{not} on this printout. Keep this first page printed for your records.

\end{abstract}

\thispagestyle{empty}
\vspace{1cm}
NAME: \line(1,0){380}
\clearpage
}

\problem{These are questions about McGrayne's book, chapter 17 and the Epilogue.}

\begin{enumerate}

\easysubproblem{What do the computer scientists who adopted Bayesian methods care most about and whose view do they subscribe to? (p233)}\spc{1}

\easysubproblem{How was \qu{Stanley} able to cross the Nevada desert?}\spc{3}

\easysubproblem{What two factors are leading to the \qu{crumbling of the Tower of Babel?}}\spc{3}

\intermediatesubproblem{Does the brain work through iterative Bayesian modeling?}\spc{4}

\easysubproblem{According to Geman, what is the most powerful argument for Bayesian Statistics?}\spc{3}

\end{enumerate}


\problem{These are questions which introduce Gibbs Sampling.}

\begin{enumerate}
\easysubproblem{Outline the systematic sweep Gibbs Sampler algorithm below.}\spc{8}


\intermediatesubproblem{We previously have shown that if $X~|~\theta \sim \binomial{n}{\theta}$ and the prior on $\theta \sim \betanot{\alpha}{\beta}}$, then $X \sim \betabinomial{n}{\alpha}{\beta}$. Even though we proved this result, pretend like you didn't know it and create a Gibbs sampler which finds $\prob{X}$.\spc{8}
\end{enumerate}


\end{document}


\problem{These are questions about McGrayne's book, chapters 15 and 16.}

\begin{enumerate}

\easysubproblem{During the H-Bomb search in Spain and its coastal regions, RAdm. William Guest was busy sending ships here, there and everywhere even if the ships couldn't see the bottom of the ocean. How did Richardson use those useless searches?}\spc{2}

\intermediatesubproblem{When the Navy was looking for the \textit{Scorpion} submarine, they used Monte Carlo methods (which we will see in class soon). How does the description of these methods by Richardson (p199) remind you of the \qu{sampling} techniques to approximate integrals we did in class?}\spc{4}

\intermediatesubproblem{What is a Kalman filter? Read about it online and write a few descriptive sentences.}\spc{4}


\intermediatesubproblem{Where do frequentist methods practically break down? (end of chapter 15)}\spc{4}

\easysubproblem{What was the main problem facing Bayesian Statistics in the early 1980's?}\spc{4}

\intermediatesubproblem{What is the \qu{curse of dimensionality?}}\spc{4}

\easysubproblem{How did Bayesian Statistics help sociologists?}\spc{4}

\easysubproblem{How did Gibbs sampling come to be?}\spc{3}

\easysubproblem{Were the Geman brothers the first to discover the Gibbs sampler?}\spc{4}

\easysubproblem{Who officially discovered the expectation-maximization (EM) algorithm? And who \textit{really} discovered it?}\spc{4}

\intermediatesubproblem{How did Bayesians \qu{break} the curse of dimensionality?}\spc{4}

\intermediatesubproblem{Consider the integrals we use in class to find expectations or to approximate PDF's / PMF's --- how can they be replaced?}\spc{4}

\easysubproblem{What did physicists call \qu{Markov Chain Monte Carlo} (MCMC)? (p222)}\spc{1}

\easysubproblem{Why is sampling called \qu{Monte Carlo} and who named it that?}\spc{4}

\easysubproblem{The Metropolis-Hastings (MH) Algorithm is world famous and used in myriad applications. Why didn't Hastings get any credit?}\spc{4}

\easysubproblem{The combination of Bayesian Statistics + MCMC has been called ... (p224)}\spc{1}


\extracreditsubproblem{p225 talks about Thomas Kuhn's ideas of \qu{paradigm shifts.} What is a \qu{paradigm shift} and does Bayesian Statistics + MCMC qualify?}\spc{8}

\easysubproblem{How did the \href{http://www.mrc-bsu.cam.ac.uk/software/bugs/}{BUGS} software change the world?}\spc{4}

\easysubproblem{Lindley said that Bayesian Statistics would win out over Frequentist Statistics because it was more logical. What in reality was the reason of its eventual victory?}\spc{4}

%\extracreditsubproblem{One of my PhD advisors, \href{https://statistics.wharton.upenn.edu/profile/563/}{Ed George} at Wharton told me that \qu{Bayesian Statistics is really `knowledge engineering.'} Is this true? Explain.}\spc{20}





\end{enumerate}



\end{document}
